\section{Literature Review}
\begin{frame}[allowframebreaks]{Literature Review}
\justifying

\begin{itemize}
    \item \citetitle{azizi2021tminer} \cite{azizi2021tminer}
    \item \citetitle{gao2019strip} \cite{gao2019strip}
    \item \citetitle{chen2021identification} \cite{chen2021identification}
    \item \citetitle{shao2021poisonedrecog} \cite{shao2021poisonedrecog}
\end{itemize}

% \textcolor{blue}{\citetitle{chen2021identification}\cite{chen2021identification}}
% \begin{itemize}
%     \item Analyzes changes in internal neurons of LSTM and computes the statistical information of samples to identify the keywords which belong to the backdoor trigger sentence.
%     \item Removes poisoned samples from dataset and retrains the model with clean data.
%     \item Requires access to training data, which may not always be available, specially if the model is outsourced.
%     \item Models typically have hundreds of thousands of neurons, so analyzing them is computationally expensive.
% \end{itemize}

% \framebreak

% \textcolor{blue}{\citetitle{shao2021poisonedrecog}\cite{shao2021poisonedrecog}}
% \begin{itemize}
%     \item Trains a preliminary model first without the embedding model, and classifies data to narrow the search scope.
%     \item Adds the embedding model later and reclassifies on the data from the first step to identify backdoors.
%     \item Requires access to training data.
% \end{itemize}

% \framebreak

% \textcolor{blue}{\citetitle{azizi2021tminer}\cite{azizi2021tminer}}
% \begin{itemize}
%     \item Probes the suspicious classifier and learns to produce text sequences that are likely to contain the Trojan trigger.
%     \item Analyzes the produced text by generative model, and determines if there is any trigger or not.
%     \item Does not require access to training data, instead uses synthetically crafted texts.
% \end{itemize}

\end{frame}