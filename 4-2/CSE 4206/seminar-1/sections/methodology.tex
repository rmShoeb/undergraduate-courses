\section{Methodology}
\begin{frame}[allowframebreaks]{Methodology}

\begin{figure}
    \centering
\begin{tikzpicture}
%Nodes
\node[processlr](vect)                      {Text/Path Embedding};
\node[io]       (input) [left=of vect]      {Input (Source Code)};
\node[processlr](prep)  [above=of vect]     {Tokenizer / AST Generator};
\node[processlr](model) [below=of vect]     {LSTM/GRU/CNN};
\node[io]       (output)[right=of vect]     {Output};
%Lines
\draw[arrow] (input) |- (prep);
\draw[arrow] (prep) --  (vect);
\draw[arrow] (vect) -- (model);
\draw[arrow] (model) -| (output);
\end{tikzpicture}
    \caption{Work-flow of a typical Source Code Vectorization method}
\end{figure}

\framebreak

\begin{figure}
    \centering
\begin{tikzpicture}
%Nodes
\node[processp] (vect)                      {Node/Edge Embedding};
\node[io]       (input) [left=of vect]      {Input (Source Code)};
\node[processp] (prep)  [above=of vect]     {Control Flow Graph (CFG) Generator};
\node[processp] (model) [below=of vect]     {Graph Neural Network (GNN)/ Convolutional Network (GCN)};
\node[io]       (output)[right=of vect]     {Output};
%Lines
\draw[arrow] (input) |- (prep);
\draw[arrow] (prep) --  (vect);
\draw[arrow] (vect) -- (model);
\draw[arrow] (model) -| (output);
\end{tikzpicture}
    \caption{Work-flow of the proposed Source Code Vectorization method}
\end{figure}

\end{frame}